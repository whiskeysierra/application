%%%%%%%%%%%%%%%%%%%%%%%%%%%%%%%%%%%%%%%%%
% "ModernCV" CV and Cover Letter
% LaTeX Template
% Version 1.1 (9/12/12)
%
% This template has been downloaded from:
% http://www.LaTeXTemplates.com
%
% Original author:
% Xavier Danaux (xdanaux@gmail.com)
%
% License:
% CC BY-NC-SA 3.0 (http://creativecommons.org/licenses/by-nc-sa/3.0/)
%
% Important note:
% This template requires the moderncv.cls and .sty files to be in the same 
% directory as this .tex file. These files provide the resume style and themes 
% used for structuring the document.
%
%%%%%%%%%%%%%%%%%%%%%%%%%%%%%%%%%%%%%%%%%

%----------------------------------------------------------------------------------------
%	PACKAGES AND OTHER DOCUMENT CONFIGURATIONS
%----------------------------------------------------------------------------------------

\documentclass[11pt,a4paper,sans]{moderncv} % Font sizes: 10, 11, or 12; paper sizes: a4paper, letterpaper, a5paper, legalpaper, executivepaper or landscape; font families: sans or roman

\usepackage[utf8]{inputenc}

\usepackage[ngerman]{babel}

\moderncvstyle{classic} % CV theme - options include: 'casual' (default), 'classic', 'oldstyle' and 'banking'
\moderncvcolor{blue} % CV color - options include: 'blue' (default), 'orange', 'green', 'red', 'purple', 'grey' and 'black'

\usepackage{lipsum} % Used for inserting dummy 'Lorem ipsum' text into the template

\usepackage[scale=0.75]{geometry} % Reduce document margins
\setlength{\hintscolumnwidth}{4cm} % Uncomment to change the width of the dates column
%\setlength{\makecvtitlenamewidth}{10cm} % For the 'classic' style, uncomment to adjust the width of the space allocated to your name

%----------------------------------------------------------------------------------------
%   NAME AND CONTACT INFORMATION SECTION
%----------------------------------------------------------------------------------------

\firstname{Willi} % Your first name
\familyname{Schönborn} % Your last name

% All information in this block is optional, comment out any lines you don't need
\title{Lebenslauf}
\address{Pieskower Weg 12}{10409 Berlin}
\mobile{+49 178 6734208}
\phone{+49 30 53085940}
\email{w.schoenborn@gmail.com}

\homepage{www.github.com/whiskeysierra}{} % The first argument is the url for the clickable link, the second argument is the url displayed in the template - this allows special characters to be displayed such as the tilde in this example

%\extrainfo{additional information}

\photo[75pt][0pt]{pictures/profile} % The first bracket is the picture height, the second is the thickness of the frame around the picture (0pt for no frame)
\quote{"Mein Fulli pusselt." -Willi Schönborn}


\begin{document}

\makecvtitle % Print the CV title

%----------------------------------------------------------------------------------------
%	PERSONAL INFORMATION SECTION
%----------------------------------------------------------------------------------------

\section{Profile}

\cvitem{Date of birth}{September 13th, 1984}
\cvitem{Nationality}{german}
\cvitem{Marital status}{married, 2 kids}

%----------------------------------------------------------------------------------------
%	WORK EXPERIENCE SECTION
%----------------------------------------------------------------------------------------

\section{Experience}

%------------------------------------------------

\cventry{07/2018 -- Today}{Principal Engineer}{}{\textsc{Zalando SE}}{Berlin}{
\begin{itemize}
    \item Software design and architecture
    \item Engineering Guidelines
    \item Maintaining OpenSource libraries
    \item Migrations to Kubernetes
    \item Decommissioning of vintage data center applications
    \item Interviewing, hiring, coaching and on-boarding
\end{itemize}}

%------------------------------------------------

\cventry{12/2015 -- 06/2018}{Senior Software Engineer}{}{\textsc{Zalando SE}}{Berlin}{
\begin{itemize}
    \item Sales Order API design/implementation
    \item Customs System Redesign and implementation
    \item Platform concept
    \item RESTful API Guidelines
    \item Migrations to AWS
    \item Maintaining OpenSource libraries
    \item Interviewing, hiring, coaching and on-boarding
\end{itemize}}

%------------------------------------------------

\cventry{03/2013 -- 11/2015}{Software Engineer}{}{\textsc{Zalando SE}}{Berlin}{
\begin{itemize}
    \item Sales Order processing using a BPMN Engine
    \item Rewrite of refund processing (processes 5+ billion euros per year)
\end{itemize}}

%------------------------------------------------

\cventry{01/2008 -- 08/2013}{Junior Software Engineer}{}{\textsc{CosmoCode GmbH}}{Berlin}{
\begin{itemize}
    \item Java/PHP web applications
    \item Agile methodologies: Scrum and Kanban
    \item Technical project management
    \item Technical concepts
    \item Customer communication
    \item Performance optimizations
    \item Support
    \item Quality assurance
\end{itemize}}

%------------------------------------------------

\cventry{04/2005 -- 06/2005}{Internship}{\textsc{Heckel \& Schulz \& Co. GmbH}}{Berlin}{}{\begin{itemize}
\item Hardware purchase and installation
\item Planning and implementation of networks
\end{itemize}}

%------------------------------------------------

\cventry{07/2004 -- 03/2005}{Civil service}{\textsc{Nordbahn gGmbH}}{Schönfließ}{}{}

%----------------------------------------------------------------------------------------
%	EDUCATION SECTION
%----------------------------------------------------------------------------------------

\section{Education}

\cventry{10/2010 -- 04/2013}{Master of Science}{Medieninformatik}{Beuth Hochschule für Technik}{Berlin}{\textit{Grade -- 1.4}}  % Arguments not required can be left empty
\cventry{10/2005 -- 09/2008}{Bachelor of Science}{Medieninformatik}{Technische Fachhochschule}{Berlin}{\textit{Grade -- 1.6}}
\cventry{09/1997 -- 06/2004}{Allgemeine Hochschulreife}{Abitur}{Marie-Curie-Gymnasium}{Hohen Neuendorf}{\textit{Grade -- 2.2}}
\cventry{08/1991 -- 07/1997}{Grundschule}{Glienicke/Nordbahn}{}{}{}

%----------------------------------------------------------------------------------------
%	PROJECTS SECTION
%----------------------------------------------------------------------------------------
\section{References}

\subsection{Riptide}
\cvitem{Type}{Open Source}
\cvitem{Year}{2015 -- 2020}
\cvitem{Source code}{\url{https://github.com/zalando/riptide}}
\cvitem{Technologies}{\textsc{Java}, \textsc{Spring Boot}, \textsc{Apache HTTP Client}}
\cvitem{Description}{Riptide is a High-Level HTTP Client that comes with a rich set of plugins, e.g. logging, tracing, metrics, resilience and more.}

\vspace{10pt}

\subsection{Logbook}
\cvitem{Type}{Open Source}
\cvitem{Year}{2015 -- 2020}
\cvitem{Source code}{\url{https://github.com/zalando/logbook}}
\cvitem{Technologies}{\textsc{Java}, \textsc{Spring Boot}, \textsc{Apache HTTP Client}, \textsc{Servlet}, \textsc{JAX-RS}}
\cvitem{Description}{Logbook is an extensible Java library to enable complete request and response logging for different client- and server-side technologies.}

\vspace{10pt}

\subsection{OpenTracing Toolbox (f.k.a. Tracer)}
\cvitem{Type}{Open Source}
\cvitem{Year}{2015 -- 2020}
\cvitem{Source code}{\url{https://github.com/zalando/opentracing-toolbox}}
\cvitem{Technologies}{\textsc{Java}, \textsc{OpenTracing}, \textsc{Spring Boot}, \textsc{Apache HTTP Client}, \textsc{Servlet}, \textsc{JDBC}}
\cvitem{Description}{OpenTracing Toolbox is a collection of libraries that build on top of OpenTracing and provide extensions and plugins to existing instrumentations.}

\vspace{10pt}

\subsection{Palava}
\cvitem{Type}{Open Source}
\cvitem{Year}{2010 -- 2012}
\cvitem{Team}{Tobias Sarnowski \& Oliver Lorenz}
\cvitem{Source code}{\url{https://github.com/palava}}
\cvitem{Technologies}{\textsc{Java}, \textsc{Guice}, \textsc{JBoss Netty}}
\cvitem{Description}{Palava is an extremely modularized application framework for Java backend. The central piece is a very powerful inter-process communication API, which allows to implement transport-agnostic endpoints which can then be accessed by a wide range of technologies, e.g \textsc{SOAP}, \textsc{REST}, \textsc{RMI}, \textsc{XML-RPC}, \textsc{JSON-RPC}, etc.}

\vspace{10pt}

\subsection{TheLabelFinder}
\cvitem{Year}{2008 -- 2013}
\cvitem{URL}{\url{http://www.thelabelfinder.com}}
\cvitem{Technologies}{\textsc{Palava}, \textsc{Java}, \textsc{PHP}, \textsc{Solr}, \textsc{MySQL}, \textsc{Apache}, \textsc{Nginx}}
\cvitem{Description}{TheLabelFinder is a fashion search engine built using Palava. The persistence, index and query capabilties are provided by MySQL and several Solr nodes which replicate in a master-slave setup.}

\newpage
%----------------------------------------------------------------------------------------
%	TECHNOLOGIES SECTION
%----------------------------------------------------------------------------------------

\section{Technologies}
Mit den folgenden Technologies arbeite ich täglich bis regelmäßig:
\newline

\cvitem{Languages}{\textsc{Java}, \textsc{SQL}, \textsc{AspectJ}, \textsc{Python}, \LaTeX, \textsc{Shell}}
\cvitem{Data Stores}{\textsc{PostgreSQL}}
\cvitem{Libraries}{\textsc{Guava}, \textsc{Lombok}}
\cvitem{Frameworks}{\textsc{Spring}, \textsc{Spring Boot}, \textsc{Camunda}}
\cvitem{Testing}{\textsc{Junit 5}, \textsc{Mockito}, \textsc{Cucumber}, \textsc{JaCoCo}}
\cvitem{Cloud}{\textsc{AWS EC2}, \textsc{AWS Cloudformation}, \textsc{Kubernetes}}
\cvitem{Logging}{\textsc{Scalyr}}
\cvitem{SCM}{\textsc{Git}}
\cvitem{Build Tools}{\textsc{Maven}}

%----------------------------------------------------------------------------------------
%	LANGUAGES SECTION
%----------------------------------------------------------------------------------------

\section{Language skills}

\cvitemwithcomment{German}{Native}{}
\cvitemwithcomment{English}{Fluent}{}

%----------------------------------------------------------------------------------------

\end{document}
